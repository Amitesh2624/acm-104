\documentclass{article}
\usepackage{graphicx}% Required for inserting images
\usepackage{lindrew}
\usepackage{pdfpages}
\usepackage[shortlabels]{enumitem}
\usepackage{matlab-prettifier}
\usepackage{algorithm}
\usepackage{algpseudocode}

\title{ACM 104 Problem Set 6}
\author{Amitesh Pandey}
\date{November 2024}
\begin{document}
\maketitle
\section*{Problem 3: Matrix Diagonalization}
\emph{Solution. } We are given
\begin{equation*}
    F = \begin{pmatrix}
        1 & 1\\
        1 & 0
    \end{pmatrix}
\end{equation*}
First, let's find the eigenvalues of $F$. We will have 
\begin{equation*}
    \det(F - \lambda I) = 0 \implies \det \begin{pmatrix}
        1 - \lambda & 1\\
        1 & -\lambda
    \end{pmatrix} = 0
\end{equation*}
This means $\lambda^2 - \lambda - 1= 0$. On solving this quadratic, we obtain values
\begin{align*}
    \lambda &= \frac{1 + \sqrt{5}}{2}\\
    &= \frac{1 - \sqrt{5}}{2}
\end{align*}
For eigenvectors $v = \langle x, y\rangle$ we will have
\begin{equation*}
    F v = \lambda v
\end{equation*}
This produces the following system of linear equations
\begin{align*}
    x + y &= \lambda x\\
    x &= \lambda y
\end{align*}
Depending on $\lambda$, we get the two eigenvectors (putting $y = 1$) as
\begin{equation*}
    v_{1} = \begin{pmatrix}
        \frac{1 + \sqrt{5}}{2}\\
        1
    \end{pmatrix},  v_{2} = \begin{pmatrix}
        \frac{1 - \sqrt{5}}{2}\\
        1
    \end{pmatrix}
\end{equation*}
\noindent{So} in the diagonalised form, $F$ simply becomes
\begin{equation*}
    F_{d} = \begin{pmatrix}
        \frac{1 + \sqrt{5}}{2} & 0\\
        0 & \frac{1 - \sqrt{5}}{2}
    \end{pmatrix}, B = \begin{pmatrix}
    \frac{1 + \sqrt{5}}{2} & \frac{1 - \sqrt{5}}{2}\\
    1 & 1
    \end{pmatrix} \implies B^{-1} = \begin{pmatrix}
        \frac{1}{\sqrt{5}} & -\frac{1-\sqrt{5}}{2 \sqrt{5}} \\
-\frac{1}{\sqrt{5}} & \frac{1+\sqrt{5}}{2 \sqrt{5}}
    \end{pmatrix}
\end{equation*}
Now we can see that 
\begin{equation*}
    F = B F_{d} B^{-1} = \begin{pmatrix}
    \frac{1 + \sqrt{5}}{2} & \frac{1 - \sqrt{5}}{2}\\
    1 & 1
    \end{pmatrix} \begin{pmatrix}
        \frac{1 + \sqrt{5}}{2} & 0\\
        0 & \frac{1 - \sqrt{5}}{2}
    \end{pmatrix}\begin{pmatrix}
        \frac{1}{\sqrt{5}} & -\frac{1-\sqrt{5}}{2 \sqrt{5}} \\
-\frac{1}{\sqrt{5}} & \frac{1+\sqrt{5}}{2 \sqrt{5}}\end{pmatrix}
\end{equation*}
\newpage
\section*{Problem 4: Symmetric Matrices}
\emph{Solution. }(a) Yes, $\text{Sym}(n)$ is a vector space, this is because the operations have the same properties on this set as they do on $\mathbb{M}_{n \times n}$, additionally, note that for any symmetric matrix, the additive inverse is guaranteed to be symmetric as well, also, the additive identity $\mathbf{0}$ is obviously symmetric. \\\\
(b) Note that the diagonals can be arbitrary in symmetric matrices. We must have $n$ matrices to account for arbitrary values on the diagonal. Additionally, we need to account for each ``flipped" position across the diagonal \emph{once}. We count all $i, j $ such that $i \neq j$. This is $n^2 - n$. Removing double counting, we get $(n^2 - n)/2$. So finally, the dimension is $(n^2 + n)/2 = n(n+1)/2$. These are the number of linearly independent matrices we need to generate all matrices in $\text{Sym}(n)$. The basis will simply be the set of all where all entries on the diagonal get a matrix with 1 on that entry and 0 elsewhere and each ``flipped" element gets an entry 1 with 0 everywhere else except for the corresponding flipped position. \\\\
\includepdf[pages=-]{PS6matlab.pdf}
\end{document}
